
\begin{frame}{Summary: the minimal workflow} 
    \vfill

    Minimal workflow
    \begin{itemize}
        \item Track the status of your project using the GitLab Kanban board.
        \item Learn and apply the very basics of git, understand the concepts behind branching, merging, commits and working with remote repositories.   
        \item Integrate the changes that worked, don't leave 30 branches open. 
        \item Format the data using established (open-source) formats if possible, if not, at least do this for secondary data (diagrams and tables). 
        \item Share your secondary data on a (TUdatalib) data repository. 
        \item Periodically cross-link research data: publication/report, scripts/code, secondary and primary data. 
    \end{itemize}

\end{frame}


%\begin{frame}{Lessons learned}
	%\vfill
        %\begin{itemize}
            %\item Keeping the workflow as simple as possible is crucial for acceptance.
            %\item Focusing on secondary data simplifies the workflow significantly.  
            %\item For simulations that run $<24$ hours primary data can be recomputed easily. 
            %\item Periodical cross-linking of research data is done quickly and it is very beneficial. 
            %\item Personal responsibility is crucial at University research groups: who are the maintainers? 
                %\begin{itemize}
                    %\item What are the incentives for maintainers? 
                %\end{itemize}
            %\item Fixing the (parallel) I/O of legacy scientific codes requires a large amount of effort. 
                %\begin{itemize}
                    %\item It should be done outside of research projects. 
                %\end{itemize}
        %\end{itemize}
%\end{frame}

%\begin{frame}{Outlook}
	%\vfill
	%\begin{itemize}
            %\item Performance CI jobs run on 64-core workstations: moving on to the HPC cluster. 
            %\item Singularity GitLab executor? 
            %\item Jupyter Hub for interactive analysis of problems in parameter variations?
            %\item Automatic publishing and cross-linking of CI artifacts? 
                %\begin{itemize}
                    %\item Source code archive, Singularity container, secondary data. 
                    %\item Data repository API must be used to modify metadata. 
                %\end{itemize}
	%\end{itemize}
%\end{frame}

