%% This is file `DEMO-TUDaBeamer.tex' version 3.10 (2021/02/22),
%% it is part of
%% TUDa-CI -- Corporate Design for TU Darmstadt
%% ----------------------------------------------------------------------------
%%
%%  Copyright (C) 2018--2021 by Marei Peischl <marei@peitex.de>
%%
%% ============================================================================
%% This work may be distributed and/or modified under the
%% conditions of the LaTeX Project Public License, either version 1.3c
%% of this license or (at your option) any later version.
%% The latest version of this license is in
%% http://www.latex-project.org/lppl.txt
%% and version 1.3c or later is part of all distributions of LaTeX
%% version 2008/05/04 or later.
%%
%% This work has the LPPL maintenance status `maintained'.
%%
%% The Current Maintainers of this work are
%%   Marei Peischl <tuda-ci@peitex.de>
%%   Markus Lazanowski <latex@ce.tu-darmstadt.de>
%%
%% The development respository can be found at
%% https://github.com/tudace/tuda_latex_templates
%% Please use the issue tracker for feedback!
%%
%% If you need a compiled version of this document, have a look at
%% http://mirror.ctan.org/macros/latex/contrib/tuda-ci/doc
%% or at the documentation directory of this package (if installed)
%% <path to your LaTeX distribution>/doc/latex/tuda-ci
%% ============================================================================
%%
% !TeX program = lualatex
%%

%% This is file `DEMO-TUDaBeamer.tex' version 3.10 (2021/02/22),
%% it is part of
%% TUDa-CI -- Corporate Design for TU Darmstadt
%% ----------------------------------------------------------------------------
%%
%%  Copyright (C) 2018--2021 by Marei Peischl <marei@peitex.de>
%%
%% ============================================================================
%% This work may be distributed and/or modified under the
%% conditions of the LaTeX Project Public License, either version 1.3c
%% of this license or (at your option) any later version.
%% The latest version of this license is in
%% http://www.latex-project.org/lppl.txt
%% and version 1.3c or later is part of all distributions of LaTeX
%% version 2008/05/04 or later.
%%
%% This work has the LPPL maintenance status `maintained'.
%%
%% The Current Maintainers of this work are
%%   Marei Peischl <tuda-ci@peitex.de>
%%   Markus Lazanowski <latex@ce.tu-darmstadt.de>
%%
%% The development respository can be found at
%% https://github.com/tudace/tuda_latex_templates
%% Please use the issue tracker for feedback!
%%
%% If you need a compiled version of this document, have a look at
%% http://mirror.ctan.org/macros/latex/contrib/tuda-ci
%% ============================================================================
%%
% !TeX program = lualatex
%%

\documentclass[
	ngerman,%globale Übergabe der Hauptsprache
	aspectratio=169,%Beamer eigene Option zum Umschalten des Formates
	color={accentcolor=2d},
	logo=true,%Kein Logo auf Folgeseiten
	colorframetitle=true,%Akzentfarbe auch im Frametitle
%	logofile=example-image, %Falls die Logo Dateien nicht vorliegen
	]{tudabeamer}
\usepackage[main=ngerman]{babel}
\usepackage{iftex}
\ifPDFTeX
\usepackage[utf8]{inputenc}%kompatibilität mit TeX Versionen vor April 2018
\fi

%Makros für Formatierungen der Doku
%Im Allgemeinen nicht notwendig!
\let\code\texttt

\title{A Workflow for Increasing the Quality of Scientific Software\\ (in Computational Science and Engineering)}
\subtitle{Exascale Computing Project (ECP) Webinar 2021-04-07}
\author[\textbf{T. Mari\'c}, JP. Lehr, I. Pappagianidis, B. Lambie, D. Bothe, C. Bischof]{Tomislav Mari\'c}
\department{TU Darmstadt, Germany}
\institute{CRC 1194 : Z-INF}

%Fremdlogo
%Logo Macro mit Sternchen skaliert automatisch, sodass das Logo in die Fußzeile passt
\logo*{\includegraphics{./crc-logo}}

% Da das Bild frei wählbar nach Breite und/oder Höhe skaliert werden kann, werden \width/\height entsprechend gesetzt. So kann die Fläche optimal gefüllt werden.
%Sternchenversion skaliert automatisch und beschneidet das Bild, um die Fläche zu füllen.
%\titlegraphic*{\includegraphics{example-image}}
\titlegraphic{\includegraphics[scale=0.47]{figures/ZINF-CI-diagram.pdf}}
\date{ECP Webinar 2021-04-07}

\usepackage{booktabs}
\usepackage{fontawesome}

\hypersetup{
  colorlinks,
  citecolor=violet,
  linkcolor=red,
  urlcolor=violet}

\begin{document}

\maketitle

\begin{frame}{Computational Science and Engineering software in\\university research groups}
	\framesubtitle{Boundary and initial conditions}
	
	\vfill
	\begin{itemize}
            \item Publish or perish \faGraduationCap\footnote{This symbol denotes a sacrifice of code quality for more papers.} prioritizes publications over scientific software.
		\item Dedicated resources for increasing software quality are usually not available.
		\item Ph.D. students rotate every ~4-5 years, postdocs every 1-2 years. 
			\begin{itemize}
				\item Little or no overlap between successors and predecessors. 
			\end{itemize}
		\item Large-scale software design is not a necessary part of the CSE curriculum. 
			\begin{itemize}
				\item Different CSE background: (Applied) Mathematics, Mechanical Engineering, Physics, Informatics.
			\end{itemize}
		\item Real-world example: onboarding people into \href{https://www.openfoam.com/documentation/guides/latest/api/classes.html}{\beamergotobutton{OpenFOAM}} module development.
	\end{itemize}
\end{frame}

\begin{frame}{Computational Science and Engineering software in\\university research groups}
	\framesubtitle{Divergence}
	
	\vfill
	\begin{itemize}
            \item Not being able to continue development from an earlier state.
            \item Reproducing results from a publication is not possible.  
                \begin{itemize}
                    \item Data, source code and publication are not archived and cross-linked. 
                    \item The version used to generate the data is not documented. 
                \end{itemize}
            \item Not being able to re-use a model from a publication. 
                \begin{itemize}
                    \item The model is not implemented in a modular way.
                    \item Version integration was not done.
                    \item Non-granular commits were used. 
                \end{itemize}
            \item Having no overview of the impact of a change on the rest of the module.
	\end{itemize}

	\medskip

\end{frame}

%\begin{frame}{CSE software in the Collaborative Research Center 1194}

    %\begin{itemize}
        %\item OpenFOAM
    %\end{itemize}

%\end{frame}

\begin{frame}{A workflow for increasing the quality of scientific CSE software} 

    \vfill
    \begin{enumerate}
        \item Track the issues in a Kanban board. 
            \begin{itemize}
                \item Work on the same issue with multiple sub-tasks together. 
                \item Model issues as \href{https://betterscientificsoftware.github.io/PSIP-Tools/PTCs/}{Progress Tracking Cards}.
            \end{itemize}
        \item Use a pragmatic version-control branching model. 
        \item Apply Test-Driven Development (TDD) for CSE software.
        \item Enable Continuous Integration with an emphasis on result visualization. 
        \item Cross-link software, result data, and report/article when reaching a milestone.
            \begin{itemize}
                \item When submitting a publication to peer-review. 
                \item After the publication has been accepted. 
                \item When giving up on an idea. 
            \end{itemize}
        \item Bonus step: publish a Singularity image with the code and data.
    \end{enumerate}
\end{frame}

\begin{frame}{A workflow for increasing the quality of (academic) CSE software} 
    \framesubtitle{OpenFOAM}

        \vfill

        The workflow is developed with OpenFOAM projects but it is tested with other software. 

        \vspace{1cm}

        \textbf{Disclaimer}: This offering is not approved or endorsed by OpenCFD Limited, producer and distributor of the OpenFOAM software via www.openfoam.com, and owner of the OPENFOAM®  and OpenCFD®  trade marks. 

\end{frame}

\begin{frame}{Pragmatic version-control branching model} 
    \framesubtitle{Separation of Concerns and Single Responsibility}

	\vfill
	\begin{itemize}

            \item University research teams \emph{working on the same project} are generally small (2 - 5 members).
            \item \href{https://en.wikipedia.org/wiki/Separation_of_concerns}{\beamergotobutton{Separation of Concerns (SC)}} and \href{https://en.wikipedia.org/wiki/Single-responsibility_principle}{\beamergotobutton{Single Responsibility Principle (SRP)}} significantly simplify the branching model. 

            \item \textbf{Separation of Concerns}: code is organized in non-overlapping layers and sections. 

            \item \textbf{Single Responsibility}: functions or classes perform single clear tasks.

            \item SC and SRP can be applied to any software.
            \item Dogmatism should be avoided: single responsibility vs less responsibilities. 
            \item OpenFOAM already uses object-oriented and generic software design patterns.  

        \end{itemize}
\end{frame}

\begin{frame}{Pragmatic version-control branching model} 
    \framesubtitle{Change integration}

        \vfill

	\begin{itemize}
            \item Keep the branching model as simple as possible.  
            \item Main and development branches are protected and managed by Maintainers. 
            \item Maintainers are responsible for git tags and cleanup: 
            \begin{itemize}
                    \item \textbf{Main}: integrations from \emph{accepted publications} and \emph{development branch}. 
                    \item \textbf{Development}: integration of \emph{broadly (CI)-tested improvements}. 
                    \item \textbf{Feature}: SRP reduces git-conflicts with researchers working on different files.
            \end{itemize}
            \item Complex branching workflow $\Rightarrow$ complications with onboarding new members.
	\end{itemize}

\end{frame}

\begin{frame}{Test Driven Development} 
    \framesubtitle{Program CSE tests first}
        \vfill


        TDD\footnote{Freeman, Steve, and Nat Pryce. Growing object-oriented software, guided by tests. Pearson Education, 2009.} for CSE
        \begin{itemize}
            \item Defining model verification and validation tests at the start.
            \item Focus placed the final result: interpolation, integration, discretization, PDE solution, physics. 
            \item Top-down, instead of bottom-up testing.
            \item Don't go overboard with unit-tests \faGraduationCap: extend unit-tests when debugging a failing CSE test.  
                \begin{itemize}
                    \item Focus placed on tests with real-world (publication) input. 
                \end{itemize}
        \end{itemize}

\end{frame}

\begin{frame}{Test Driven Development} 
    \framesubtitle{The CSE tests the class / function interface}
        \vfill

    \begin{itemize}
        \item \textbf{New code}: it is easier to program the API you wish for, if you are its first user. 
            \begin{itemize}
                \item Make the class interface easy to use correctly and difficult to use incorrectly\footnote{Scott Meyers. 2014. Effective Modern C++: 42 Specific Ways to Improve Your Use of C++11 and C++14 (1st. ed.). O'Reilly Media, Inc.}.
                \item Reduce number of function arguments, single responsibility, clear naming, ... 
            \end{itemize}
        \item \textbf{Legacy code}: extend existing API without modification. 
            \begin{itemize}
                \item OpenFOAM: understanding class hierarchies, \textit{finding a base class with Runtime Type Selection and a virtual function to overload.}
            \end{itemize}
        \item \textbf{The test application is the solver application with a different input.}
            \begin{itemize}
                \item If possible, testing and solution is done by the same code.  
                \item This prevents code duplication. 
                \item Data output and additional checks can be disabled by (compile-time) options.
            \end{itemize}
    \end{itemize}

\end{frame}


\begin{frame}{Test Driven Development} 
    \framesubtitle{Jupyter notebooks}

    \vfill
    Jupyter notebooks 
    \begin{itemize}
        \item \textbf{Documentation}: geometry, initial and boundary conditions, error norms, comparison data.
        \item \textbf{Processing}: conservation, convergence, stability, validation errors. 
        \item \textbf{Analysis}: live, interactive and and remote result analysis for long simulations.
    \end{itemize}
\end{frame}

\begin{frame}{Test Driven Development} 
    \framesubtitle{Parameter tests: Jupyter notebooks}
    
    \vfill
    \begin{center}
            \includegraphics[width=0.9\textwidth]{figures/Cluster-Parameter-Study-Organization.pdf}
    \end{center}
\end{frame}

\begin{frame}{Test Driven Development} 
    \framesubtitle{Parameter tests: primary data (simulation results) organization}

    \begin{itemize}
        \item The quality of CSE software is measured against verification and validation data. 
        \item Automatic testing and comparison with previous versions (and other works) hinges on data organization and formats.
    \end{itemize}
    
    \vfill
    \begin{itemize}
        \item \textbf{Legacy code primary data organization}: 
            \begin{itemize}
                \item use the existing folder structure and parameterization tools \faGraduationCap,
                \item make sure the mapping 'caseXYZ' $\to$ 'parameter vector' is stored (YAML, ...)
            \end{itemize}
        \item \textbf{New code primary data organization}: 
            \begin{itemize}
                \item Like legacy code: a folder structure and 'caseXYZ' $\to$ 'parameter vector' mapping (YAML, ...) \faGraduationCap.
                \item HDF5\footnote{\url{https://www.hdfgroup.org/solutions/hdf5}} or other open data format.
                \item Alternative: \textbf{ExDir}\footnote{Dragly, Svenn-Arne, et al. 'Experimental Directory Structure (Exdir): An alternative to HDF5 without introducing a new file format.' Frontiers in neuroinformatics 12 (2018): 16.} 
            \end{itemize}
    \end{itemize}

\end{frame}


\begin{frame}{Test Driven Development} 
    \framesubtitle{Parameter tests: secondary data (tables and diagrams) organization}

    \texttt{pandas.MultiIndex} CSV with metadata for secondary data
    \begin{itemize}
        \item \texttt{pandas.MultiIndex} saved in "metadata columns". 
        \item \textcolor{red}{\textbf{Metadata is repeated}}: not an issue for the tiny secondary data! 
        \item Metadata in columns $\to$ \texttt{pandas.MultiIndex} $\to$ trivial input for data analysis. 
        \item \textbf{Direct readable export of tables to LaTex!}
    \end{itemize}

    \footnotesize

    \begin{tabular}{lllrrr}
        \toprule
        {} &         H &     L\_INF &  O(L\_INF) &  EPSILON\_R\_EXACT\_MAX &  O(EPSILON\_R\_EXACT\_MAX)  \\ 
        VELOCITY\_MODEL &           &           &           &                      &                        \\ 
        \midrule
        \textcolor{red}{\textbf{SHEAR\_2D}}       &  0.125000 &  0.032961 &  1.833407 &             0.032961 &                1.833407 \\ 
        \textcolor{red}{\textbf{SHEAR\_2D}}       &  0.062500 &  0.009249 &  1.955529 &             0.009249 &                1.955529 \\ 
        \textcolor{red}{\textbf{SHEAR\_2D}}       &  0.031250 &  0.002385 &  1.988745 &             0.002385 &                1.988745 \\ 
        \textcolor{red}{\textbf{SHEAR\_2D}}       &  0.015625 &  0.000601 &  1.997178 &             0.000601 &                1.997178 \\ 
        \textcolor{red}{\textbf{SHEAR\_2D}}       &  0.007813 &  0.000150 &  1.999294 &             0.000150 &                1.999294 \\ 
        \textcolor{red}{\textbf{SHEAR\_2D}}       &  0.003906 &  0.000038 &  1.999294 &             0.000038 &                1.999294 \\ 
        \bottomrule
    \end{tabular}

\end{frame}

\begin{frame}{Continuous Integration with result visualization} 
	\framesubtitle{Schematic diagram}

	\centering
	\includegraphics[width=0.8\textwidth]{figures/ZINF-CI-diagram.pdf}

\end{frame}

\begin{frame}{Continuous Integration with result visualization} 
    \framesubtitle{Testing machine, Docker images, artifacts}

\end{frame}

\begin{frame}{Continuous Integration with result visualization} 
    \framesubtitle{A GitLab runner with a docker executor and a local image}

\end{frame}

\begin{frame}{Continuous Integration with result visualization} 
    \framesubtitle{Building}

\end{frame}

\begin{frame}{Continuous Integration with result visualization} 
    \framesubtitle{Running}

\end{frame}

\begin{frame}{Continuous Integration with result visualization} 
    \framesubtitle{Visualization}

\end{frame}

\begin{frame}{Cross-linking data, source code and reports/publications} 
	\framesubtitle{Schematic diagram}
	
	\begin{center}
		\includegraphics[width=0.57\textwidth]{figures/cross-linking.pdf}
	\end{center}

\end{frame}

\begin{frame}{Cross-linking data, source code and reports/publications} 
	
	\vfill
	\begin{itemize}
		\item Archive secondary data on a data repository (DOI). 
		\item Archive primary data on a data repository (DOI). 
		\item Create a git tag on the remote git repository. 
		\item Archive the source code, binaries and secondary data in a Singularity container (DOI). 
		\item Refer to result data DOIs, the git repo and the tag in the publication.
		\item Upload the publication to a repository (DOI, arXiv-ID).
			\begin{itemize}
				\item For tech reports and milestones, DOIs can be requested without publication.
			\end{itemize}
		\item Edit metadata and the git-tag description until everything is cross-linked.
	\end{itemize}

\end{frame}

\begin{frame}{Similarity with published workflows / practices}

	\vfill
	Our \emph{(subjective)} estimates* of similarity $1-5$ (higher is more similar), $-$: aspect not addressed.
	\begin{center}
		\scriptsize
		\begin{tabular}{@{} *6l @{}}    \toprule
				\emph{DOI} & \emph{Branching model} & \emph{TDD} & \emph{Cross-linking} & \emph{CI}  & (Meta)data standardization \\\midrule
				 \href{https://doi.org/10.12688/f1000research.11407.1}{10.12688/f1000research.11407.1} 
					 & -  & -  & -  & - & 1  \\ 
				 \href{https://doi.org/10.3934/math.2016.3.261}{10.3934/math.2016.3.261} 
					 & -  & -  & -  & - & 2  \\ 
				 \href{https://doi.org/10.1371/journal.pbio.1001745}{10.1371/journal.pbio.1001745} 
					 & 1  & 2  & -  & - & -  \\ 
				 \href{https://doi.org/10.1371/journal.pcbi.1005510}{10.1371/journal.pcbi.1005510}
					 & -  & -  & 3 & 1 & 3  \\ 
				 \href{https://doi.org/10.1145/2723872.2723881}{10.1145/2723872.2723881}
					 & 1  & -  & - & 1 & -  \\ 
				 \href{https://dl.acm.org/doi/10.1145/3324989.3325719}{10.1145/3324989.3325719}
					 & 1  & -  & - & 5 & -  \\ 
				 \href{https://doi.org/10.1371/journal.pone.0230557}{10.1371/journal.pone.0230557}
					 & 1  & -  & - & 1 & 4  \\ 
				 \href{https://doi.org/10.1145/3219104.3219147}{10.1145/3219104.3219147} 
					 & 1  & -  & -  & 4 & - \\\bottomrule
				 \hline
		\end{tabular}
	\end{center}
	
	*\emph{The list may still be incomplete.}
	
\end{frame}

\begin{frame}{Outlook}
	\vfill
	\begin{itemize}

                \item A more detailed webinar: \href{https://www.exascaleproject.org/event/workflow4scisoft}{https://www.exascaleproject.org/event/workflow4scisoft}.


		\item Automatic publication and cross-linking of CI artefacts. 
			\begin{itemize}
				\item Source code archive, Singularity container, secondary data. 
				\item Data repository API must be used to modify metadata. 
			\end{itemize}
		\item Jupyter Hub for interactive analysis of problems in parameter variations.
		\item Performance CI jobs run on 64-core workstations: moving to the HPC cluster. 
	\end{itemize}
\end{frame}

\begin{frame}{Acknowledgements}

	\vfill
	{\
		Funded by the German Research Foundation (DFG) – Project-ID 265191195 – SFB 1194 : Z-INF
	}

\end{frame}

\end{document}

